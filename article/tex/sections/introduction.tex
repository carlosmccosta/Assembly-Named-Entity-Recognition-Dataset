\section{Introduction}\label{sec:introduction}

To achieve these goals, the robot needs to successfully recognize the objects within its workspace and semantically track their pose with high precision while the operator demonstrates how to perform the assembly operations. This type of teaching allows rapid reprogramming of flexible robotic assembly cells for new tasks. This process can be speed up further if there are assembly manuals available, which allows the robotic system to extract the objects and their assembly spacial disposition from the textual and visual representations. By knowing which objects to expect for a given teaching session, the object recognition efficiency can be significantly increased (by limiting the object search database). Moreover, this preliminary information extraction phase reduces the human teaching phase to only the operations that lack detailed information.
